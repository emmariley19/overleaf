\subsection{Mechanisms}
In this section I examine the various mechanisms at work in this paper. I first look at the mechanism through which mobile money facilitates consumption smoothing after an aggregate shock, arguing that this is via remittances flows. I then break down the impact of mobile money by distance to the nearest mobile money agent and between urban and rural areas. I look in more detail into what is driving the negative impact of rainfall shocks by separating the impact of the shock on consumption by droughts and floods. I use a measure of rainfall as the variation from the mean to characterise non-linearities in the effect of rainfall on consumption as justification for using a 1 standard deviation as my definition of a shock. I confirm that rainfall shocks are exogenous by checking that neither of the shock variables is correlated with household demographic variables.  Finally, I examine some other aggregate shocks reported in the TNP survey to expand my results.  

\subsubsection{Remittances}
The proposed mechanism tested in this paper is that mobile money allows remittances to be sent by friends and family in other locations in response to an aggregate shock and that this allows consumption smoothing. However, it is possible that mobile money affects consumption smoothing in other ways. One possible alternative is that mobile money allows funds to be safely stored on a mobile phone as savings which can be run down in response to a shock. A second is that households might be considered more creditworthy if they use mobile money and are able to borrow more when an adverse event happens. In the TNP survey, 80\% of respondents said they send and receive money as the most important reason for using mobile money. In the third round of the survey, questions were asked on who sent remittances, by what channel, from where and what their relation was to the recipient. 40\% of remittances were sent by a son or daughter, 35\% were sent via mobile money and 30\% came form Dar es Salaam, the capital city. This suggests a story of children sending remittances home from the city via mobile money is a plausible one.  

In order to test whether remittances are driving the way that mobile money protects against adverse shocks I run the following specification: 
\begin{align}
r_{jv}=& C_{jv} + \gamma_a AggShock_{jv} + \mu MM_{jv} \notag \\ 
& + \beta_m MM_{jv}\cdot AggShock_{jv} + \bm{\theta X_{jv}} + \varepsilon_{jv} 
\end{align}
where $r_{jv}$ is the remittance amount received by a mobile money using household and the other variables are as defined previously. Log consumption per capita, $C_{jv}$, is included to control for income effects. Unfortunately, data on remittance amounts is only available for the final wave of the panel and so this specification can only be run as an OLS regression for one period. It still gives an indication though whether remittances are responding to negative shocks. If remittances are the channel through which mobile money smooths aggregate shocks then the following prediction will hold:
\begin{description}
\item{\bf{Prediction 6}} $\gamma_a>0$
\end{description}
so that remittances increase in response to an aggregate shock.   

\begin{table}
\centering 
\caption{OLS regression of remittances received after an aggregate shock} \label{remittance}
\begin{tabular}{lcc}
\multicolumn{3}{c}{Dependent variable: remittances received (Tanzania Shilling)} \\ \hline
 & (1) & (2)  \\
  & 1 sd shock & self reported shock \\ \hline
  &  &   \\
Rain shock & 34,786*** & 66,486 \\
 & (15,298) & (46,823) \\
MM use & 27,456** & 19,767* \\
 & (11,708) & (11,196) \\
Rain shock *MM use & -23,240  & 66,486  \\
 & (36,306) & (46,392)  \\
Observations & 3,311 & 3,311  \\
R-squared & 0.06 & 0.07 \\ \hline
\multicolumn{3}{p{11cm}}{Full set of control variables as in Table \ref{HH sum} and errors clustered at the village level. All regressions also control for village characteristics which could affect the ease of sending remittances. These are the distance to the nearest main road, distance to nearest population centre and distance to nearest market. MM use is a dummy variable equal to one if the household used mobile money in a given year.} \\
\multicolumn{3}{l}{ Standard errors in brackets, *** p$<$0.01, ** p$<$0.05, * p$<$0.1} \\
\end{tabular}
\end{table}
Table \ref{remittance} shows the OLS regressions of remittances received in Tanzanian Shillings in wave 3 for both self reported shocks and the 1 standard deviation rainfall shock. Remittances increase after a negative shock by 32,000-45,600 shilling (\$17-24) and this is significant at the 1\% level. Mobile money use results in the households receiving more remittances overall compared to other forms of receiving remittance \footnote{55\% of households send remittances via friends or family and 46\% via mobile money}, by between 25,500 and 46,000 shilling (\$13-25). The interaction of mobile money with the shock is not significant in either regression and is actually slightly negative in regression (2) (by 2000 shilling, \$1) though with a very large standard error. A potential reason for the high standard errors is that there are only around 100 households who reported the amount of remittances they received, experienced a rainfall shock and use mobile money  and so this result is more imprecisely estimated on a small sample\footnote{1500 households received remittances, 200 of these also experienced a rainfall shock and 100 of these use mobile money}. The results however are indicative that remittances are driving the mechanism through which mobile money smooths consumption. 

\subsubsection{The impact of distance to the mobile money agent}
The distance to the nearest mobile money agent could also impact how easy it is for someone to send and receive remittances via mobile money and hence the benefit they receive from using this service. I therefore run specification \eqref{eq: specification agg shock} with interactions with dummy variables for whether an agent is within 1km of the village, between 1km and 5km, between 5km and 10km away, with agents more than 10km away as the exclusion category. This will also indicate whether the distance to the nearest mobile money agent changes the pattern of sharing remittances within the village or the receiving household keeping the remittances. For example, it might be easier to hide remittances the further away the mobile money agent is from the village. 

The results for distance dummies interacted with each variable are reported in Table \ref{distance}. I find that having other households in the village using mobile money increases a households per capita consumption if the mobile money agent is located within 1km of the village. Since on average 1/3 of the village use mobile money if anyone does then this corresponds to an increase of 9\% per capita, a large effect. In regression (2) this result also holds for agents within 5km of the village. When a rainfall shock occurs, there is a small negative effect of other households in the village using mobile money in regression (1), but otherwise the coefficients on the shock interaction with village mobile money use are insignificant. Households using mobile money themselves benefit by between 11 and 20\% of per capita consumption when a rainfall shock occurs, more than cancelling out the negative impact of the shock of 7\%. However this effect is only strong if an agent is within 1km for regression (2) and significant at the 10\% level for agents within 5km in regression (1). 

These findings indicate that both benefit to others in the village and benefit to the user during a rainfall shock decrease the further away the mobile money agent is, with most benefits if the agent is within 1km of the village. I do not find evidence that households are sharing less with the village, and instead keeping the remittance for themselves, as the distance to the nearest mobile money agent increases. This could be because household members are less willing to walk a long distance to a mobile money agent, giving up other productive activities to do so and so are using mobile money less the further away the agent is. This could be tested if I had detailed data on the number and size of remittances received, but unfortunately I don't.   

\begin{table}
\centering \caption{Interactions with distance to the nearest mobile money agent} \label{distance}
\def\arraystretch{0.80}
\begin{tabular}{lcc} \hline
 & (1) & (2) \\
 & self reported drought or flood & 1sd rainfall shock \\ \hline
Rain shock & -0.078*** & -0.073*** \\
& (0.031) & (0.025) \\
village MM (agent 1km) & 0.252*** & 0.279*** \\
 & (0.065) & (0.067) \\
village MM (agent 5km) & 0.151 & 0.284**  \\
  & (0.156)  & (0.145) \\
village MM (agent 10km) & 0.249 & 0.047 \\
 & (0.201) & (0.203) \\
Shock*village MM (agent 1km)  & -0.016*** &  -0.009 \\
 & (0.006) &  (0.007)  \\
Shock*village MM (agent 5km) & 0.041 & -0.030 \\
 & (0.035) & (0.033)  \\
Shock*village MM (agent 10km)  & 0.036 & -0.013 \\
 & (0.060) &  (0.031) \\
MM use (agent 1km)  & 0.030 & 0.039 \\
 & (0.036) & (0.040) \\
MM use (agent 5km) & 0.050 &  0.078  \\
 & (0.085) &  (0.096)  \\
MM use (agent 10km)  & 0.094 & -0.005 \\
 & (0.095) & (0.113) \\
shock*MM use (agent 1km) & 0.204*** &  0.112**  \\
 & (0.068) &  (0.014) \\
shock*MM use (agent 5km) & 0.201* & 0.218 \\
 & (0.113) & (0.150) \\
shock*MM use (agent 10km) & 0.066 &  0.205   \\
 & (0.177) &  (0.178) \\
Observations & 8,367 & 8,367 \\
Number of households & 2,936 & 2,936 \\
R-squared & 0.287 & 0.267 \\
  \hline
\multicolumn{3}{p{14cm}}{All regressions include a full set of household control variables from table \ref{HH sum} and errors clustered at the village level.  All regressions also control for village characteristics which could affect the ease of sending remittances. These are the distance to the nearest main road, distance to nearest population centre and distance to nearest market, as well as having an ATM, bank or post office in the village.Village MM use refers to the proportion of households in the village using mobile money. MM use is a dummy variable equal to one if that household uses mobile money. The distance to the agent is a dummy variable equal to 1 if the agent is within that distance of the village.} \\
\multicolumn{3}{l}{Standard errors in brackets, *** p$<$0.01, ** p$<$0.05, * p$<$0.1} \\
\end{tabular}
\end{table}

\subsubsection{Urban-rural differences}
I run the results separately for urban and rural households to see if there are differential effects for these groups. Mobile money services would be expected to benefit rural households more since they have less access to other ways to send remittances such as banks or designated money transfer shops (such as Weston Union) and are less likely to have friends or relative passing by regularly who could bring remittances. They are also more reliant on agriculture and so affected by rainfall shocks more directly in terms of crop losses than households in urban areas. 

Separate urban and rural results are shown in in Table \ref{urban rural} where columns (1) and (3) show results for rural areas and columns (2) and (4) are for urban areas. From the results it can be seen that rainfall shocks have larger and more significant effects in rural areas than in cities, most likely because rural areas are reliant on agriculture for income making their consumption more sensitive to rainfall. The benefit from other people using mobile money in the community seems to come largely from rural areas, where with 1/3 of the village using mobile money every household benefits by 10\% of per capita consumption. In urban areas this effect is only 3\% and not significant in regression (4). When there is a rain shock, some of the benefit of having other mobile money using households in your community disappear, though this is not significant in any of the regressions.  

\begin{table}
\centering
\caption{Fixed effects regressions by urban and rural  households} \label{urban rural}
\begin{tabular}{lcccc}
\multicolumn{5}{c}{Dependent variable: Log consumption per capita} \\ \hline
& \multicolumn{2}{c}{Self reported shock} & \multicolumn{2}{c}{1sd rain shock} \\
 & (1) & (2) & (3) & (4) \\
 & rural & urban& rural & urban  \\ \hline
 &  &  &  &  \\
Rain shock & -0.07** & -0.02 &   -0.10*** & -0.07*   \\
 & (0.03) & (0.06)  & (0.04) & (0.04)  \\
Village MM use & 0.30*** & 0.11** & 0.33*** & 0.11 \\
 & (0.11) & (0.05) & (0.09) & (0.07) \\
Rain shock*village MM use & -0.20 & -0.00 & -0.16 & 0.10  \\
 & (0.16) & (0.11) & (0.13) & (0.09)  \\
MM use & -0.02 & 0.04 & -0.03 & 0.07* \\
 & (0.04) & (0.03) & (0.04) & (0.05) \\
Rain shock*MM use & 0.11* & 0.07 & 0.19** & 0.01  \\
 & (0.07) & (0.06) & (0.08) & (0.05)  \\
 &  &  &  &  \\
Observations & 6,529 & 3,241 & 6,529 & 3,241 \\
Number of households & 3,570 & 1,854 & 3,570 & 1,854 \\
R-squared & 0.10 & 0.27 & 0.10 & 0.28 \\\hline
\multicolumn{5}{p{11cm}}{Full set of control variables as in table \ref{HH sum}, household fixed effects and errors clustered at the village level. Village MM use refers to the proportion of households in the village using mobile money. All regressions also control for village characteristics which could affect the ease of sending remittances. These are the distance to the nearest main road, distance to nearest population centre and distance to nearest market, as well as having an ATM, bank or post office in the village. Mobile money use is a dummy variable equal to one if that household uses mobile money.} \\
\multicolumn{5}{l}{ Standard errors in brackets, *** p$<$0.01, ** p$<$0.05, * p$<$0.1} \\
\end{tabular}
\end{table}

Using mobile money yourself is only significant in urban areas in regression (4), giving your household an extra 6.6\% of consumption per capita. However when there is a rainfall shock it is people in rural areas who largely benefit, getting between 11 and 19\% of per capita consumption. The lack of benefit in urban areas could be because shocks are not having a significantly negative effect to begin with due to the lack of dependence of income on rainfall, or because people in cities are better protected from rainfall shocks in other ways such as from having better quality infrastructure that, for instance, prevents roads being washed away during heavy rain. These results confirm that rural households have more to gain than urban ones from increased access to mobile money services due to a combination of less methods for obtaining remittances and greater reliance on rain dependent agricultural as an income source. 

\subsubsection{Droughts and floods}
To see whether too much or too little rainfall have differential effects on consumption and the ability of mobile money to smooth these impacts, I separate out the effects of droughts compared to floods. A drought is defined as the difference in rainfall from the mean being more than one standard deviation below the mean and a flood as the difference in rainfall from the mean being more than one standard deviation above the mean. This is reported in Table \ref{Drought flood}, where droughts are reported in columns (1)-(2) and floods in columns (3)-(4), each as a simple difference-in-difference and with household fixed effects. 

It can be clearly seen that it is mainly droughts which have a significant negative effect of around 10\% of per capita consumption. Floods have no significantly different effect from zero. I perform an F test of the equality of the drought and flood effects. Equality is rejected at the 1\% level  for the simple difference-in-difference specification but only at the 10\% level for the regression with household fixed effects. Hence there is some evidence for differential effects of droughts and floods, with droughts primarily negatively affecting consumption. Having other people in the village use mobile money is significant in all the regressions. At the mean of 1/3 of people using mobile money in the village, consumption per capita is between 7 and 9\% higher. Mobile money use is only significant in the simple difference-in-difference specification, increasing consumption per capita by 13-15\%, but this effect goes to 5\% once household fixed effects are accounted for. Looking at the shock interactions, other people in the village using mobile money during a shock is not significant in any of the regressions. The shock interacted with a household's own mobile money use is significant in both of the drought regressions at the 10\% level. 

These results confirm my earlier findings but reveal that it is only droughts that actually cause a reduction in per capita consumption. Mobile money increases per capita consumption after a rainfall shock only when a drought  occurs, not when a flood occurs, which makes sense since floods aren't causing a drop in consumption to begin with. These results also link with the earlier finding that it is mainly rural households who experience a drop in consumption after a rainfall shock and benefit most from mobile money. Rural households are likely to be most affected by a drought since it directly reduces their income through crop losses, whereas urban households will only be affected indirectly. These results therefore support each other. The lack of impact of floods might also indicate the presence of non-linearities in the impact of rainfall, with too much rain initially increasing crop yields, whereas too little rain always has a negative impact on crop yields. I examine possible non-linearities in more detail below.

\begin{table}
  \centering
  \caption{Differential impacts of droughts and floods} \label{Drought flood}
\begin{tabulary}{1\textwidth}{Lcccc} \multicolumn{5}{c}{Dependent variable: Log consumption per capita} \\\hline
& \multicolumn{2}{c}{Drought} & \multicolumn{2}{c}{Flood} \\ \cmidrule(r){2-3} \cmidrule(l){4-5}
 & (1) & (2) & (3) & (4)  \\
VARIABLES & Diff-in-diff & FE  & Diff-in-diff & FE \\ \hline
 &  &  &  &    \\
Shock & -0.10*** & -0.11*** & 0.02 & -0.01   \\
 & (0.03) & (0.03) & (0.03) & (0.03)  \\
Village MM use & 0.12** & 0.23***  & 0.10* & 0.27**  \\
 & (0.05) & (0.04)  & (0.06) & (0.13) \\
Shock*village MM use & 0.19 & 0.07  & -0.01 & -0.06   \\
 & (0.16) & (0.17)  & (0.08) & (0.08)  \\
MM use & 0.15*** & 0.05*  & 0.13*** & 0.05*  \\
 & (0.02) & (0.02)  & (0.03) & (0.03)  \\
Shock*MM use & 0.10* & 0.14*  & 0.06 & 0.01  \\
 & (0.06) & (0.08)  & (0.04) & (0.05)  \\
Observations & 9,770 & 9,770  & 9,770 & 9,770\\
Number of households  & 3,798 & 3,798 & 3,798 & 3,798 \\
R-squared & 0.57 & 0.13  & 0.55 & 0.08  \\ \hline

 F stat on equality & 7.19 &  3.36 & & \\
 p value & 0.008 &  0.067 & & \\
 \hline
\multicolumn{5}{p{12cm}}{All regression include a full set of control variable from Table \ref{HH sum}, time dummies and clustered errors at the village level. Village MM use refers to the proportion of households in the village using mobile money. All regressions also control for village characteristics which could affect the ease of sending remittances. These are the distance to the nearest main road, distance to nearest population centre and distance to nearest market, as well as having an ATM, bank or post office in the village. Mobile money use is a dummy variable equal to one if that household uses mobile money. The F stat is a test of equality of the drought and flood coefficients for the diff-in-diff and fixed effects regressions. The resulting p value is reported as well. } \\
\multicolumn{5}{l}{ Standard errors in brackets, *** p$<$0.01, ** p$<$0.05, * p$<$0.1} \\
\end{tabulary}
\end{table}

\subsubsection{Non-linearities in rainfall shocks}
Rainfall shocks have very non-linear effects, with a little more rain often good for crops but only extreme amounts harmful, and likewise a little less rain unlikely to cause much hardship whereas a drought would cause widespread crop damage. I use deviations of rainfall from the mean to check for non-linear effects of the rainfall shock magnitude, and this is reported in Table \ref{Rain dev}. In this table I look at the deviations of rainfall from the mean and the square of this to find any turning points in the impact of rainfall on per capita consumption. Column (1) shows the results for both positive and negative deviations, whereas column (2) only looks at deviations below the mean and column (3) at deviations above the mean. 

It can be seen in column (2) that negative shocks always have a negative effect on consumption, with 200mm less rain in the year reduced consumption by 9.2\%. More rainfall initially has a positive effect on log consumption per capita but this becomes negative for rainfall more than 300mm above the mean, with 500mm more rainfall than usual reducing log consumption by 4.2\%. This shows that while droughts always have a negative impact on per capita consumption, more rain initially has a positive impact and it is only sufficiently extreme rainfall that has a negative effect. 
\input{rain_dev}

\subsubsection{Exogeneity of rainfall shock}
I confirm the exogeneity of the rainfall shock by showing that the rainfall shock is not correlated with any characteristics of the households. Fixed effect regressions for each of the rainfall shocks are shown in Table \ref{shock corr}. In column (1), the 1 standard deviation rainfall shock, only using a ROSCA is significant at the 10\% level. For the self reported rainfall shocks in column (2), ROSCA is again significant at the 10\% level but has the opposite sign from the regression in column (1). The agriculture household head occupation dummy is also significant at the 10\% level with working in agriculture associated with a 9\% higher probability of reporting a shock, possibly because the household is more sensitive to the weather if they work in agriculture. Overall these results suggest my rainfall shock variables are not correlated with characteristics of the households. 


\begin{table}
\centering \caption{Shock correlations} \label{shock corr}
\def\arraystretch{0.7}
\begin{tabular}{lcc} \hline
 & (1) & (2) \\
 & Self reported drought/flood  & 1sd rain shock \\ \hline
Rural& -0.027** & -0.046 \\
 & (0.013) &(0.048) \\
Wealthscore & 0.004*** & 0.002 \\
 & (0.001) & (0.012) \\
Head education & -0.001 & 0.029*** \\
 & (0.002) & (0.011)  \\
Head age & -0.003 & 0.001 \\
 & (0.002) & (0.002) \\
Household size & 0.005  & 0.023** \\
 & (0.004) & (0.010) \\
Mobile phone & 0.012 & 0.077 \\
 & (0.013) & (0.077)\\
Loan & 0.022 & -0.127 \\
 & (0.015) & (0.101) \\
Bank account  & -0.026**  & -0.089\\
 & (0.012)  & (0.100)  \\
ROSCA & 0.026 & -0.547*** \\
 & (0.030)  & (0.172)  \\
Agriculture & 0.045 & 0.230  \\
 & (0.033) & (0.180)  \\
Fishing  & 0.062 & 0.049  \\
 & (0.041) & (0.273) \\
Government  & 0.030 & 0.024 \\
 & (0.049)& (0.232)  \\
Parastatal & 0.008 & -0.717 \\
 & (0.057)  & (0.476) \\
Private sector  & -0.006 & 0.177 \\
 & (0.027) & (0.212) \\
NGO/Religious & -0.034 & -0.053\\
 & (0.056)  & (0.431) \\
Self employed & 0.001 & 0.202\\
 & (0.025)  & (0.203) \\
Family work& -0.008  & 0.644*\\
 & (0.029)  & (0.355) \\
Unemployed & 0.013 & 0.262\\
 & (0.037) & (0.736)  \\
Observations & 9,281 & 1,090 \\
R-squared & 0.056 & 0.114 \\ \hline
\multicolumn{3}{p{12cm}}{Regression one has village fixed effects and is at the village level for village averages of the household characteristics, since the one standard deviation rainfall shock is constant for a village. Regression two has household fixed effects and village clustered errors and is at the household level since each household reports whether it experienced a shock or not.  } \\
\multicolumn{3}{l}{ Standard errors in brackets, *** p$<$0.01, ** p$<$0.05, * p$<$0.1} \\
\end{tabular}
\end{table}
 
\subsubsection{Other shocks}
Finally I check whether the results found are particular to floods and droughts or also extend to other aggregate shocks. The theoretical predictions I made based on equation \eqref{eq: specification agg shock} apply to any aggregate shock, not only rainfall shocks, and so I check that my results generalise by running this specification again with different aggregate shock measures. These results are reported in Table \ref{other shocks}. I have self-reported data on three other aggregate shocks: Large fall in crop prices, large rise in food prices and large rise in agricultural input prices. 

None of the three shocks have a significant negative effect on per capita consumption. However, when a shock occurs, those using mobile money themselves experience a 16\% increase in per capita consumption in two of the regressions despite the shock not causing a fall in consumption. Regression (2) may not show a significant effect because 40\% of people in the sample reported experiencing large rises in food prices so this shock affects large numbers of people across the country at the same point in time, making it less likely someone not experiencing this shock could support someone experiencing it. Other people in the village using mobile money still has a highly significant 8\% increase on per capita consumption ($1/3$ of 17\%), similar to that found before. 

Overall then, these results provide support that my main findings using rainfall extend to any aggregate shocks, not just rainfall shocks, but suggest it might be rainfall shocks which households currently have the most difficulty smoothing themselves. 

\begin{table}
\centering \caption{Diff-in-diff fixed effect regressions of other aggregate shocks on consumption} \label{other shocks}
\begin{tabular}{lccc} 
\multicolumn{4}{c}{Dependant variable: Log consumption per capita}\\ \hline
 & (1) & (2) & (3) \\
 & Large fall crop price & Large rise food price & Large rise agri input price \\ \hline
Shock & 0.09*** & 0.01  & 0.07***  \\
 & (0.02) & (0.03)  & (0.03)  \\
Village MM use & 0.10* & 0.06 & 0.11** \\
 & (0.05) & (0.06) & (0.05) \\
Shock*village MM & -0.24*** & 0.07   & -0.25  \\
 & (0.13)  & (0.07)  & (0.10) \\
Mobile money use & 0.04 & 0.05 & 0.03 \\
 & (0.02) & (0.03) & (0.03) \\
Shock*MM use & 0.10* & -0.01  & 0.15***
  \\
 & (0.07) & (0.04)  & (0.05) \\

Observations & 7,704 & 7,704 & 7,704 \\
R-squared & 0.178 & 0.176 & 0.178 \\
Number of households & 3,032 & 3,032 & 3,032 \\
  \hline
Mean of User & 0.118 & 0.118 & 0.118 \\
 Mean of Shock & 0.123 & 0.423 & 0.121 \\ \hline
\multicolumn{4}{p{14cm}}{ All regressions include full set of household control variables from Table \ref{HH sum}, household fixed effects and errors clustered at the village level. All regressions also control for village characteristics which could affect the ease of sending remittances. These are the distance to the nearest main road, distance to nearest population centre and distance to nearest market. Village MM use refers to the proportion of households in the village using mobile money. Mobile money use is a dummy variable equal to one if that household uses mobile money. } \\
\multicolumn{4}{l}{ Standard errors in brackets, *** p$<$0.01, ** p$<$0.05, * p$<$0.1} \\
\end{tabular}
\end{table}

\subsection{Discussion} 
This paper is a contribution to the early literature on mobile money because it both confirms an earlier finding that mobile money helps households smooth  shocks (Jack \& Suri 2014) and extends this analysis to the wider impact on the rest of the community from having mobile money users in the village. Looking at just the impact on the user of mobile money would therefore underestimate the benefits of having mobile money in a community. Benefits to both the user and community are highest in rural areas and decrease sharply with distance to the nearest mobile money agent. Efforts to facilitate the spread of mobile money agents to more communities would increase welfare, especially since the more remote the community the less likely the community has other ways to receive remittance and so the larger the benefits from having mobile money services could be. 

The fact that mobile money increases average village consumption but only helps the recipient household smooth consumption after an aggregate shock is an important contribution to the risk sharing literature. Exploring why this is the case would advance our understanding of how informal risk sharing arrangements operate in developing countries. For example, are households choosing to hide remittances or are they opting out entirely from the risk sharing network in the village? Looking at transfers within the village and from a migrant in response to idiosyncratic shocks would allow these effects to be separated. My data was not detailed enough on specific network patterns and remittance flows to answer this question, but the answer is important for understanding how new technologies change traditional risk sharing patterns and for a deeper understanding of how risk sharing networks are sustained.  

There are many potential explanations for why remittances might not be shared after an aggregate shock. \cite{coate1993} looked at risk sharing as a repeated game and determined the conditions under which an informal risk sharing network is self-sustaining. To be self-sustaining, at every point in time the benefits from remaining in the relationship must outweigh the benefit of defection. These conditions include the ability to punish  deviations from risk sharing which requires information on network members. In regards to mobile money, information on the amount of money received may be hard for other villagers to obtain, particularly if the mobile money agent is outside the village itself. Without accurate information on if remittances have been received and the amount, it is very hard for the rest of the villagers to ask for more transfers when a shock has occurred or to punish the mobile-money-using household for not sharing remittances. Differences in the impact of an aggregate shock on each individual household also creates uncertainty for households to know the extend to which other households' consumption falls after a village level shock. Again this means other households in the village cannot be sure that a mobile money using household is smoothing its consumption due to remittances received or because it simply suffered less from the aggregate shock by chance. 

Secondly, the benefit from defection might increase for a mobile-money-using household, because they can then keep any remittances received for themselves at every point in time. They could insure both aggregate and idiosyncratic shocks through the migrant, and the benefits of this risk sharing relationship might exceed those of the village risk sharing relationship. In this case, mobile money using households might no longer participate at all in village risk sharing. However, it is unclear then why I find that having a household using mobile money in the village  benefits every other households if the mobile-money-using household is not sharing remittances at all.  

Another reason why remittances aren't shared after an aggregate shock is that norms for sharing in the village don't extend to aggregate shock. Norms develop to address situations which an individual cannot solve alone but that the village as a whole can overcome. Traditionally, the norm has been that only individual shocks are shared by the village. Aggregate shocks were not shared because they impacted the entire village at once and so could not be smoothed. The introduction of mobile money has enabled households to have links with other households in other locations and hence for some households in a village to smooth aggregate shocks. However the social norm might still required households to only share income for idiosyncratic shocks, not aggregate. Hence there is no requirement for households to share income after an aggregate shock, explaining why I don't see sharing in the data. However there is a norm that households pool their income when an aggregate shock has not occurred, resulting in mobile-money-using households sharing remittances and overall village consumption being higher.

\cite{coate1993} also showed that after a big event like a famine in which incomes fall substantially, the discount factor required for a household to remain in a risk sharing arrangement goes up as the income of the others goes down. This results in the break down of informal risk sharing relationships if the contribution of a relative fortunate household is insufficient to keep the other households from starvation. It is possible that a similar situation occurs after an aggregate shocks, with some households experiencing larger income falls than others and risk sharing breaking down temporarily.  

The literature on risk sharing has often found that risk sharing isn't necessarily taking place at the village level but instead in sub-networks within the village and with networks of family and friends outside the village (Fafchamps \& Lund 2003, De Weerdt \& Dercon 2006). If this is the proper network over which risk sharing takes place, I might not find sharing at the village level to aggregate shocks when in fact there is risk sharing occurring within sub-networks that extend beyond the village. It therefore appears as though mobile money users are keeping remittances themselves after an aggregate shock when in fact they are sharing across villages. It would be an interesting piece of further work to examine in detail the networks of mobile money users and non users to find out how they overlap and address this question. 

A final reason for lack of sharing of remittances after an aggregate shock could be due to pressure from the migrant to prevent the receiving household from sharing any additional remittances with other households in the village. If the migrant is already sending a steady stream of remittances to the household in the village and is then asked to send more because of a rainfall shock, they might only send more remittances conditional on these not being shared with the rest of the village. The migrant is not in the risk sharing network with the rest of the village and so has no desire to have additional transfers shared with the entire village. I also assumed that the migrant knew accurately the consumption of the receiving household and the average shortfall due to an aggregate shock. Hence the migrant would only send enough to allow that one household to smooth consumption, not enough to share with the entire village.  

While all of these explanations are possible, it would be interesting future work to examine them in detail and determine which are the strongest effects. This links into the question of determining how, if at all, improved access to remittances changes traditional risk sharing networks in villages, for better and for worse. 



