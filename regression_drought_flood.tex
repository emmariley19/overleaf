\begin{table}
  \centering
  \caption{Differential impacts of droughts and floods} \label{Drought flood}
\begin{tabulary}{1\textwidth}{Lcccc} \multicolumn{5}{c}{Dependent variable: Log consumption per capita} \\\hline
& \multicolumn{2}{c}{Drought} & \multicolumn{2}{c}{Flood} \\ \cmidrule(r){2-3} \cmidrule(l){4-5}
 & (1) & (2) & (3) & (4)  \\
VARIABLES & Diff-in-diff & FE  & Diff-in-diff & FE \\ \hline
 &  &  &  &    \\
Shock & -0.10*** & -0.11*** & 0.02 & -0.01   \\
 & (0.03) & (0.03) & (0.03) & (0.03)  \\
Village MM use & 0.12** & 0.23***  & 0.10* & 0.27**  \\
 & (0.05) & (0.04)  & (0.06) & (0.13) \\
Shock*village MM use & 0.19 & 0.07  & -0.01 & -0.06   \\
 & (0.16) & (0.17)  & (0.08) & (0.08)  \\
MM use & 0.15*** & 0.05*  & 0.13*** & 0.05*  \\
 & (0.02) & (0.02)  & (0.03) & (0.03)  \\
Shock*MM use & 0.10* & 0.14*  & 0.06 & 0.01  \\
 & (0.06) & (0.08)  & (0.04) & (0.05)  \\
Observations & 9,770 & 9,770  & 9,770 & 9,770\\
Number of households  & 3,798 & 3,798 & 3,798 & 3,798 \\
R-squared & 0.57 & 0.13  & 0.55 & 0.08  \\ \hline

 F stat on equality & 7.19 &  3.36 & & \\
 p value & 0.008 &  0.067 & & \\
 \hline
\multicolumn{5}{p{12cm}}{All regression include a full set of control variable from Table \ref{HH sum}, time dummies and clustered errors at the village level. Village MM use refers to the proportion of households in the village using mobile money. All regressions also control for village characteristics which could affect the ease of sending remittances. These are the distance to the nearest main road, distance to nearest population centre and distance to nearest market, as well as having an ATM, bank or post office in the village. Mobile money use is a dummy variable equal to one if that household uses mobile money. The F stat is a test of equality of the drought and flood coefficients for the diff-in-diff and fixed effects regressions. The resulting p value is reported as well. } \\
\multicolumn{5}{l}{ Standard errors in brackets, *** p$<$0.01, ** p$<$0.05, * p$<$0.1} \\
\end{tabulary}
\end{table}