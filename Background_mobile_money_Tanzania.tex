\subsection{Context: Mobile money in Tanzania}

There are 4 mobile money providers in Tanzania ; Vodacom's M-Pesa, Zantel's Z-Pesa and Zain's Zap (now Airtel Money), all of which launched in 2008/9, and Tigo's Tigo Pesa which launched in 2010. M-pesa is by far the largest of these with 72\% of the market. Take-up of mobile money took off slowly, with only 0.5\% of households having ever used mobile money in 2009 \citep{FSD2013}, but after Vodacom initiated some changes at the end of 2009 the service took off, reaching a quarter of the population by the end of 2011 and a third by the end of 2013. From only 900 agents in September 2009, the service had 17,000 by December 2013.

Mobile money requires the user to have a mobile phone and sim card from the mobile money provider. The user must register for a mobile money account and can then deposit money through that mobile money providers' agents, which are usually located in shops. The cash is then electronically deposited in the customer's account. Customers can transfer money via SMS to other people even on different networks,  and make withdrawals at their network's agents anywhere in the country. Users are charged a step-tariff rate for sending money and for withdrawing money from agents, with fees for M-Pesa of around 10\% for withdrawing and 3\%  for sending \$5 and falling with the amount. Depositing money on the account is free.  