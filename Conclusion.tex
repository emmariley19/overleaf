\subsection{Summary}
Even if idiosyncratic shocks are shared perfectly within the village, households in developing countries are still subject to large changes in consumption due to aggregate shocks, such as rainfall shocks, which negatively affect the consumption of the entire village at one. Droughts and floods are a major source of risk to developing households, and measures which help protect against these, ranging from social protection to micro-insurance, are key areas of research. Mobile money services are a new and fast growing technology which can also help households insure their consumption against aggregate shocks by providing better access to remittances.

In this paper I show that idiosyncratic shocks are perfectly insured at the village level but that large rainfall shocks negatively affect household consumption, therefore confirming the predictions of the Mace (1991) model. I then look at the impact mobile money use has on consumption in communities with mobile money, comparing users and non-users. I extend the Mace model to include mobile money as remittances and I make predictions for the impact of aggregate shocks based on different models of remittance sharing. I find that consumption per capita is higher for all households in communities with mobile money. However, when it comes to smoothing aggregate shocks, only mobile money users are  able to smooth their consumption after the shock. 

I confirm the robustness of my findings using a placebo test from before the introduction of mobile money to test the common trends assumption, instrumental variable techniques to address potential self-selection effects into mobile money use and propensity score matching to overcome selection bias based on observable covariates. All of these confirm my results. 

I check the proposed mechanism that the consumption smoothing effect is due to households that use mobile money receiving remittances, by looking at the value of remittances received in the final wave of data. I also extend my results by looking at urban and rural households and droughts and floods separately. I find that it is principally rural mobile-money-using households that benefit from improved ability to smooth rainfall shocks and who also share remittances with their communities when an aggregate shock hasn't occurred, while urban households keep remittances for themselves. It is principally droughts which negatively affect consumption and that are smoothed when a household uses mobile money. I also find that my results are only true when there is a mobile money agent within 1-5km of the community, suggesting that the better access households have to mobile money the greater are the benefits. I also confirm my results extend to other types of aggregate shock. This contributes to our understanding of which groups benefit the most from being able to smooth aggregate shocks using remittances.   

\subsection{Discussion} 
This paper is a contribution to the early literature on mobile money because it both confirms an earlier finding that mobile money helps households smooth  shocks (Jack \& Suri 2014) and extends this analysis to the wider impact on the rest of the community from having mobile money users in the village. Looking at just the impact on the user of mobile money would therefore underestimate the benefits of having mobile money in a community. Benefits to both the user and community are highest in rural areas and decrease sharply with distance to the nearest mobile money agent. Efforts to facilitate the spread of mobile money agents to more communities would increase welfare, especially since the more remote the community the less likely the community has other ways to receive remittance and so the larger the benefits from having mobile money services could be. 

The fact that mobile money increases average village consumption but only helps the recipient household smooth consumption after an aggregate shock is an important contribution to the risk sharing literature. Exploring why this is the case would advance our understanding of how informal risk sharing arrangements operate in developing countries. For example, are households choosing to hide remittances or are they opting out entirely from the risk sharing network in the village? Looking at transfers within the village and from a migrant in response to idiosyncratic shocks would allow these effects to be separated. My data was not detailed enough on specific network patterns and remittance flows to answer this question, but the answer is important for understanding how new technologies change traditional risk sharing patterns and for a deeper understanding of how risk sharing networks are sustained.  

There are many potential explanations for why remittances might not be shared after an aggregate shock. \cite{coate1993} looked at risk sharing as a repeated game and determined the conditions under which an informal risk sharing network is self-sustaining. To be self-sustaining, at every point in time the benefits from remaining in the relationship must outweigh the benefit of defection. These conditions include the ability to punish  deviations from risk sharing which requires information on network members. In regards to mobile money, information on the amount of money received may be hard for other villagers to obtain, particularly if the mobile money agent is outside the village itself. Without accurate information on the amount of remittance received, it is very hard for the rest of the villagers to ask for more transfers when a shock has occurred. It is also hard for the village to punish the mobile-money-using household for not sharing remittances as the other village household cannot be sure the mobile-money-using household has received any additional remittances to help smooth the shock. This difficulty in knowing the quantity of remittances received makes it hard for the village to punish a household for keeping the remittances itself, and therefore is a possible explanation for why a mobile-money-using household does not share remittances after a shock and instead uses it to smooth its own consumption. Differences in the impact of an aggregate shock on each individual household also creates uncertainty for village households to know the extend to which other households' consumption falls after a shock. Again this means other households in the village cannot be sure that a mobile money using household is smoothing its consumption due to remittances received or because it simply suffered less from the aggregate shock by chance. 

Finally, the benefit from defection might increase for a mobile-money-using household, because they can then keep any remittances received for themselves at every point in time. They could insure both aggregate and idiosyncratic shocks through the migrant, and the benefits of this risk sharing relationship might exceed those of the village risk sharing relationship. In this case, mobile money using households might no longer participate at all in village risk sharing. However, it is unclear then why I find that having a household using mobile money in the village  benefits every other households if the mobile-money-using household is not sharing remittances.  

Another reason why remittances aren't shared after an aggregate shock is the norms for sharing in that village. Norms develop to address situations which an individual cannot solve alone but that the village as a whole can overcome. Traditionally, the norm has been that only individual shocks are shared by the village. Aggregate shocks were not shared because they impacted the entire village at once and so could not be smoothed. The introduction of mobile money has enabled households to have links with other households in other locations and hence for some households in a village to smooth aggregate shocks. However the social norm still required households to only share income for idiosyncratic shocks, not aggregate. Hence there is no requirement for households to share income after an aggregate shock, explaining why I don't see sharing in the data. However there is a norm that households pool their income when an aggregate shock has not occurred, resulting mobile-money-using households sharing remittances and overall village consumption being higher.

The lack of a norm for sharing during aggregate shocks can also be linked back to the repeated game explanation above. \cite{coate1993} also showed that after a big event like a famine in which incomes fall substantially, the discount factor required for a household to remain in a risk sharing arrangement goes up as the income of the others goes down. This results in the break down of informal risk sharing relationships if the contribution of a relative fortunate household is insufficient to keep the other households from starvation. It is possible that a similar situation occurs after an aggregate shocks, with some households experiencing larger income falls than others and risk sharing breaking down temporarily.     
 
The literature on risk sharing has often found that risk sharing isn't necessarily taking place at the village level but instead in sub-networks within the village and with networks of family and friends outside the village (Fafchamps \& Lund 2003, De Weerdt \& Dercon 2006). If this is the proper network over which risk sharing takes place, I might not find sharing at the village level to aggregate shocks when in fact there is risk sharing occurring within sub-networks that extend beyond the village. It therefore appears as though mobile money users are keeping remittances themselves after an aggregate shock when in fact they are sharing across villages. It would be an interesting piece of further work to examine in detail the networks of mobile money users and non users to find out how they overlap and address this question. 

A final reason for lack of sharing of remittances after an aggregate shock could be due to pressure from the migrant to prevent the receiving household from sharing any additional remittances with other households in the village. If the migrant is already sending a steady stream of remittances to the household in the village and is then asked to send more because of a rainfall shock, they might only send more remittances conditional on these not being shared with the rest of the village. The migrant is not in the risk sharing network with the rest of the village and so has no desire to have additional transfers shared with the entire village. I also assumed that the migrant knew accurately the consumption of the receiving household and the average shortfall due to an aggregate shock. Hence the migrant would only send enough to allow that one household to smooth consumption, not enough to share with the entire village.  

While all of these explanations are possible, it would be interesting future work to examine them in detail and determine which are the strongest effects. This links into the question of determining how, if at all, improved access to remittances changes traditional risk sharing networks in villages, for better and for worse. 


\subsection{Future research}

One question brought up by my paper is how the use of mobile phones to send remittances over long distances substitutes for other forms of risk sharing, including; informal credit, microfinance loans, transfers within the village via tradition risk sharing and formal rainfall insurance products. \cite{karlanudry2012} find that in Ghana the binding constraint farmer's face is uninsured risk against catastrophic loss, whereas liquidity constraints are not as large as expected. This suggests that mobile money services, by offering insurance against catastrophic shocks, could alleviate a binding constraint in places where no aggregate insurance mechanism is available. Insurance against aggregate shocks is complementary to informal risk sharing networks that only insure against farm-specific losses, as found by \cite{mobarakrosenzweig2012} looking at rainfall insurance products and caste risk sharing networks. Mobile money is a potential substitute for rainfall insurance products, which have seen low and slow take up, and expanding access to mobile money to rural areas could help mitigate the risk that the poor face of uninsured, extensive losses. Examining how mobile money services substitute for and complement other financial products and services would help our understanding of what constraints the poor face and how best to alleviate them. 

There are also many interesting further areas for research on mobile money services. One area would be to look at how mobile money affects intra-household decision making. Mobile money services have the potential to give women more control over their money by providing them with a safe place to store savings, as credit on the phones, and by allowing them to easily send and receive money with friends and relatives. Storing money on a mobile phone offers a safe saving location which a woman's husband cannot easily gain access to. Women can therefore save up for things they value and achieve more bargaining power within the household. Since women often move away from their home village to live with their husbands when they marry, mobile money services provide a new way for women to remain connected to their home village network. This gives them another source of funding for things they value or to use in times of need, potentially increasing the women's Pareto weight within the household.

Another area of interest for future work is the impact mobile money services can have on businesses. Many business now take mobile money payments for purchases and even pay employees via mobile phone. On-going research by Callen, Blumenstock and Ghani is looking at the potential for mobile money payment of salaries to reduce corruption and improve firm performance in Afghanistan. In Tanzania, utility companies allow customers to pay their bills via mobile money services. Examining these alternative uses of mobile money services and their impacts on consumers, businesses and governments is another wide area for future work. 

Finally, as mobile money services have expanded, new services are being offered. For example, M-Pesa in Kenya is now offering accounts linked to a user's bank account to help facilitate easy savings. These accounts pay interest, have no balance limits and offer emergency credit and insurance facilities. Hence new service offerings through mobile phones could dramatically increase the banking services available to the poor, with all the benefits attached to this. 

While current work has primarily used panel data or natural experiments to try to determine the causal impact of mobile money services, randomised controlled trials that explicitly look at consumption smoothing and remittance flows after shocks would help to validate these findings. Though randomised control trials are hard to carry out in this context, the need to train agents in how to provide mobile money services presents opportunities to randomly treat villages by providing training for agents and encouraging households in those villages to sign up for mobile money accounts. 

These and other issues remain to be explored in what promises to be an exciting area of future research. 

