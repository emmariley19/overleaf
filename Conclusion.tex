Even if idiosyncratic shocks are shared perfectly within the village, households in developing countries are still subject to large changes in consumption due to aggregate shocks, such as rainfall shocks, which negatively affect the consumption of the entire village at one. Droughts and floods are a major source of risk to developing households, and measures which help protect against these, ranging from social protection to micro-insurance, are key areas of research. Mobile money services are a new and fast growing technology which can also help households insure their consumption against aggregate shocks by providing better access to remittances.

In this paper I show that idiosyncratic shocks are perfectly insured at the village level but that large rainfall shocks negatively affect household consumption, therefore confirming the predictions of the Mace (1991) model. I then look at the impact mobile money use has on consumption in communities with mobile money, comparing users and non-users. I extend the Mace model to include mobile money as remittances and I make predictions for the impact of aggregate shocks based on different models of remittance sharing. I find that consumption per capita is higher for all households in communities with mobile money. However, when it comes to smoothing aggregate shocks, only mobile money users are  able to smooth their consumption after the shock. 

I confirm the robustness of my findings using a placebo test from before the introduction of mobile money to test the common trends assumption, instrumental variable techniques to address potential self-selection effects into mobile money use and propensity score matching to overcome selection bias based on observable covariates. All of these confirm my results. 

I check the proposed mechanism that the consumption smoothing effect is due to households that use mobile money receiving remittances, by looking at the value of remittances received in the final wave of data. I also extend my results by looking at urban and rural households and droughts and floods separately. I find that it is principally rural mobile-money-using households that benefit from improved ability to smooth rainfall shocks and who also share remittances with their communities when an aggregate shock hasn't occurred, while urban households keep remittances for themselves. It is principally droughts which negatively affect consumption and that are smoothed when a household uses mobile money. I also find that my results are only true when there is a mobile money agent within 1-5km of the community, suggesting that the better access households have to mobile money the greater are the benefits. I also confirm my results extend to other types of aggregate shock. This contributes to our understanding of which groups benefit the most from being able to smooth aggregate shocks using remittances.   




