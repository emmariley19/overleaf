\newpage
\section{Empirical analysis}
\subsection{Linking the model and data}
In this section I explain how I implement the estimating equations \eqref{eq: idioshock} and \eqref{eq: specification agg shock} using my household panel data set. 

To control for household characteristics, all regressions include the full sets of controls from Table \ref{HH sum} and clustered errors at the village level unless stated otherwise. Standard errors are clustered at the village level since mobile money agents are located by village and so the decision to use mobile money will be correlated within villages but not across villages. The data is also weighted in all regressions by the inverse of the probability that the observation is included in the survey. The survey was stratified in order to to produce estimates for different sub populations, for example between rural and urban households, with similar confidence intervals. The weights take this into account.

The interaction term with the shock variable, $\bm{X_{jvt}} \cdot AggShock_{jvt}$, controls for any changes in observable household characteristics which might impact the household's ability to smooth shocks. It can be seen from Table \ref{HH sum} that many of the demographic variables changed over time including education, mobile phone ownership and loans which all increased across the three waves. These could help a household smooth shocks, for example by mobile phone ownership providing access to information about shocks which makes it easier for households to smooth shocks. Including a set of covariates and interactions of these covariates with the shock controls for any effects of these variables on consumption smoothing.  

There are two self-selection effects with regards to mobile money which could bias my results. The first is self-selection by an individual to use mobile money. Self-selection effects into using mobile money are absorbed into the coefficient $\mu$ on $MM_{jvt}$, which is not the focus of my analysis. The summary statistics (Table \ref{HH adoption}) show that mobile money users are richer and better educated than non-users. Looking at the average marginal effect from a logistic regression of household characteristics on mobile money use (Table \ref{MM use access}), column (1) shows that being wealthier, owning a mobile, having more loans, having a bank account and being part of a ROSCA all increase the probability of a household using mobile money. For example, a rural household is 14\% less likely to us mobile money. Household head age and household size decrease the probability of using mobile money while education of the household head has no significant impact. The wave 3 dummy is highly significant and households are 19\% more likely to use mobile money in year 3. I control for all these variables in all the regressions. I also control for unobservable characteristics of individuals by using fixed effects. Unobservable characteristics such as technology preference or ability would otherwise bias the coefficient on mobile money use upwards. For example, people more willing to try new technology might use mobile money and also have higher incomes, biasing upwards the coefficient on mobile money use. Or poorer (richer) people might live in places that experience more shocks and use mobile money so that they can receive remittances from family elsewhere, resulting in a downwards (upwards) bias on the shock mobile money interaction. Panel data allows me to control for constant characteristics of the households.  


\begin{table}
\centering
  \caption{Correlations of mobile money use } \label{MM use}

\begin{tabulary}{1\textwidth}{lC} \hline
 & MM use=1 if household uses Mobile money \\ \hline
Rural & -0.063*** \\
 & (0.012) \\
 Wealthscore & 0.0038***  \\
 & (0.001)  \\
Head education & 0.002 \\
 & (0.001) \\
Head age& -0.001*** \\
 & (0.000) \\
Household size & -0.004*** \\
 & (0.002) \\
Mobile phone ownership & 0.176*** \\
 & (0.019) \\
Number of loans & 0.044*** \\
 & (0.009) \\
Bank account & 0.063*** \\
 & (0.009) \\
ROSCA  & 0.027 \\
 & (0.017) \\
Observations & 9,278 \\
 \hline
\multicolumn{2}{p{10cm}}{Average marginal effects from logit regressions of mobile money use and mobile money agent access. Errors are clustered at the village level and covariates are at the household level} \\
\multicolumn{2}{l}{ Standard errors in brackets, *** p$<$0.01, ** p$<$0.05, * p$<$0.1} \\
\end{tabulary}
\end{table}

\begin{table}
\centering
\resizebox{!}{0.5\textheight}{\begin{minipage}{\textwidth}
\centering
  \caption{Correlations of mobile money access } \label{MM access}
  \def\arraystretch{0.5}
\begin{tabulary}{0.7\textwidth}{lc} \hline
 & MM agent=1 if mobile money agent  within 5km \\ \hline
 &  \\
Rural & -0.036 \\
 & (0.035) \\
Base wealth & -0.009 \\
 & (0.010) \\
 Head educucation\_family & 0.013 \\
 & (0.009) \\
Head age & -0.000 \\
 & (0.002) \\
Household sizxe & -0.004 \\
 & (0.008) \\
Mobile phone ownership & -0.077 \\
 & (0.065) \\
Number of loans & 0.079 \\
 & (0.090) \\
Bank account & -0.051 \\
 & (0.075) \\
 ROSCA & 0.205 \\
 & (0.132) \\
ATM & 0.090* \\
 & (0.054) \\
Bank & -0.034 \\
 & (0.055) \\
Community tap water & -0.020 \\
 & (0.025) \\
Court & 0.015 \\
 & (0.037) \\
Dip tank & 0.037 \\
 & (0.039) \\
health\_centre & 0.019 \\
 & (0.023) \\
hospital & 0.003 \\
 & (0.045) \\
primary school & -0.005 \\
 & (0.032) \\
secondary school & 0.054** \\
 & (0.022) \\
Market daily & 0.044* \\
 & (0.026) \\
Market weekly & 0.088*** \\
 & (0.026) \\
Police station & 0.057** \\
 & (0.026) \\
Post office & -0.038 \\
 & (0.044) \\
Nursary & 0.007 \\
 & (0.035) \\
SACCOs & 0.004 \\
 & (0.023) \\
Observations & 1,060 \\ \hline
\multicolumn{2}{p{12cm}}{ The regression is run at the village level, with village level averages of the household covariates and also includes village level characteristics which could affect an agent's decision on whether to locate there. } \\
\multicolumn{2}{l}{ *** p$<$0.01, ** p$<$0.05, * p$<$0.1} \\
\end{tabulary}
\end{minipage}}
\end{table}

Secondly there is self-selection by mobile money agents into villages. If mobile money agents are more likely to select into villages with certain individual citizen characteristics, such as wealthier inhabitants, this could confound my results. For example, if mobile money agents only locate in wealthier places then this will bias the coefficient on mobile money use upwards. I therefore run a logistic regression of the presence of a mobile money agent within 5km on observed characteristics of both the village inhabitants and village characteristics.  In Table \ref{MM use access} column (2) I show the average marginal effect of each covariate. None of the coefficients are significant except for households having a bank account, at the 10\% significance level. The dummy for wave 3 is highly significant, indicating that mobile money agents are 27\% more likely to be present in a village by wave 3, a good check since mobile money access was expanding during this time. 

Overall, this suggests that mobile money agents are not choosing to select into villages based on household and village characteristics. This makes sense since the majority of agents are existing, single-owned, small businesses who previously sold airtime and sim cards for a mobile operator. Agents take a commission on the transactions they make so it benefits businesses regardless of their location to sign up to become an agent, and the mobile operators will train them, provide the equipment and a starter float in order to make it as easy as possible for a business to become an agent. Since most agents operate out of an existing business (usually a small shop) there is little movement of agents to, for example, wealthier locations, though there is a higher density of agents in wealthier and more populated locations such as cities. 


I assume aggregate shocks are exogenous, a reasonable assumption since in self reported data shocks are unexpected large events and in the rainfall constructed data they are large unusual events one standard deviation away from the mean in absolute value. I check rainfall shocks are exogenous by regressing the shock measures on the full set of covariates (see Mechanisms section) and find that the control variables do not predict a rainfall shock (for example poorer people do not experience more shocks).  

\subsection{Main results}
This section begins by looking at the results of village risk sharing of idiosyncratic shocks. I then examine the impact of aggregate shocks on users of mobile money, non-users and non-users within villages with other mobile money users.  

\subsubsection{Village risk sharing of idiosyncratic shock}
Table \ref{idio} shows OLS fixed effect regressions of different idiosyncratic shocks on per capita consumption. These were the shocks reported in the household survey to affect predominantly this household only, for example a major illness of, or loss of employment by, a household member. The regressions include village-time dummies to control for village level variations of aggregate resources, household fixed effects to remove unobserved factors affecting household consumption smoothing ability and a vector of household demographic variables to account for other household differences which could affect consumption smoothing and to improve precision. 

The results shows that none of the individual shocks cause a drop in consumption, with three of the coefficients actually positive, though nothing is significant. I therefore cannot reject the hypothesis that idiosyncratic shocks are insured perfectly by the village. 

\begin{table}[ht]
\centering
\caption{Household fixed effects regressions of different idiosyncratic shocks on consumption} \label{idio}
\begin{tabulary}{1\textwidth}{lCCCCC}
\multicolumn{6}{c}{Dependent variable: Log consumption per capita} \\\hline
 & (1) & (2) & (3) & (4) & (5)  \\
 & Household business failure & Loss of salaried employment & Chronic/ severe illness & Death of household member & Fire in home    \\ \hline
 &  &  &  & &   \\
 Shock & 0.004 & 0.083 & 0.049  & -0.015 & -0.007   \\
 & (0.077) & (0.103)  & (0.085) & (0.041) & (0.083)  \\
Observations & 9,770 & 9,770 & 9,770 & 9,770 & 9,770 \\
Number of individuals  & 3,798 & 3,798 & 3,798 & 3,798 & 3,798 \\
 R-squared & 0.558 & 0.558 & 0.558 & 0.558 & 0.558 \\ \hline
\multicolumn{6}{p{15cm}}{All regression include a full set of control variable from Table \ref{HH sum}, household fixed effects, village-time dummies and clustered errors at the village level.} \\
\multicolumn{6}{l}{ Standard errors in brackets, *** p$<$0.01, ** p$<$0.05, * p$<$0.1} \\
\end{tabulary}
\end{table}

\subsubsection{Impact of mobile money use}
Table \ref{MM spill} shows the primary results of this paper. It shows regression results of the impact of aggregate shocks on consumption for mobile money users and non-users in villages with and without other mobile money user, as in equation \eqref{eq: specification agg shock}. The first 3 columns show results using self reported droughts or floods, whereas the final three columns show results using the measure of a rainfall shock as a greater or less than 1 standard deviation difference from the mean. 
\begin{table}
\begin{adjustwidth}{-1in}{-1in} 
  \centering
  \caption{Impact of rainfall shocks on consumption for mobile money users and non-users} \label{MM spill}
\begin{tabular}{lcccccc}
\multicolumn{7}{c}{Dependent variable: Log consumption per capita} \\\hline
& \multicolumn{3}{c}{ Self-reported shock} & \multicolumn{3}{c}{1 sd rainfall shock} \\ \cmidrule(r){2-4} \cmidrule(l){5-7}
 & (1) & (2) & (3)  & (4) & (5) & (6)  \\

  & Diff-in-diff & FE & FE & Diff-in-diff & FE & FE \\ \hline
 &  &  &  &  &  &  \\
Rain shock & -0.049* & -0.046* & 0.005   & -0.042** & -0.043** & 0.060 \\
  & (0.026) & (0.024) & (0.157) & (0.020) & (0.020) & (0.067) \\
Village MM use  & 0.095* & 0.094* & 0.098* & 0.055 & 0.105 & 0.133** \\
 & (0.052) & (0.056) & (0.056) & (0.055) & (0.066) & (0.060) \\
Shock*village MM use & -0.071 & -0.104 & -0.213** & 0.074 & 0.007 & -0.044 \\
& (0.096) & (0.081) & (0.096)   & (0.074) & (0.074) & (0.079)  \\
Mobile money use & 0.137*** & -0.002 & -0.003 & 0.116*** & 0.004 & 0.005 \\
 & (0.023) & (0.023) & (0.023) & (0.024) & (0.024) & (0.025) \\
Shock*MM use   & 0.037 & 0.107** & 0.083* & 0.083** & 0.049 & 0.076* \\
 & (0.047) & (0.045) & (0.049) & (0.040) & (0.038) & (0.043) \\

Observations & 9,281 & 9,281 & 9,281 & 9,281 & 9,281 & 9,281 \\
Number of households  & 3,807 & 3,807 & 3,807 & 3,807 & 3,807 & 3,807 \\
R-squared & 0.568 & 0.195 & 0.200 & 0.568 & 0.196 & 0.205 \\ \hline
Mean of village MM use  & 0.118 & 0.118 & 0.118 & 0.118 & 0.118 & 0.118 \\
Mean of village MM use if $>0$ & 0.359 & 0.359 & 0.359 & 0.359 & 0.359 & 0.359 \\
Mean of user & 0.181 & 0.181 & 0.181 & 0.181 & 0.181 & 0.181 \\
Mean of Shock & 0.128 & 0.128 & 0.128 & 0.237 & 0.237 & 0.237 \\
Interactions with shock &  &  & YES &  &  & YES \\ \hline
\multicolumn{7}{p{15cm}}{All regressions include full set of household control variables from Table \ref{HH sum} and errors clustered at the village level. All regressions also control for village characteristics which could affect the ease of sending remittances. These are the distance to the nearest main road, distance to nearest population centre and distance to nearest market. Village MM use refers to the proportion of households in the village using mobile money. Mobile money use is a dummy variable equal to one if that household uses mobile money. Interactions are the control variables from table \ref{HH sum} interacted with the shock.} \\
\multicolumn{7}{l}{ Standard errors in brackets, *** p$<$0.01, ** p$<$0.05, * p$<$0.1} 
\end{tabular}
 \end{adjustwidth}
 \end{table}
 

Columns (1) and (4) are simple difference-in-difference regressions, columns (2) and (5) include household fixed effect  and columns (3) and (6) are fixed effects regressions including interaction terms of all the control variables with the rainfall shock. Columns (3) and (6) therefore need to be interpreted with care. The coefficient on the rain shock term will be quite different from the regressions without the shock interaction terms because this is the coefficient for when all of the covariates are zero, which is never true in the data. To find the total effect of the rainfall shock  would require using the mean value of each of the covariates and multiplying this by the coefficient on their interaction with the shock and summing these together. I can also calculate the mean shock impact for different groups, for example for urban households compared to rural, rich and poor households or mobile money user and non-users. For example, at the mean of covariates for non-mobile money users, the self reported rainfall shock has a $-0.08$ impact significant at the 1\% level. For mobile money users the shock at the mean level of covariates has a $-0.09$ impact before I include the mobile money use interaction with the shock, but after including the interaction with mobile money the total shock impact is $0.04$ meaning that the benefit mobile money users receive from remittances more than compensates for the negative rainfall shock effect.  Similar effects are seen using the 1 standard deviation rainfall shock, with the shock impact at the average covariates causing a significant 5\% drop in consumption and mobile money use more than compensating for this and resulting in a 5\% increase in per capita consumption.  The regressions including interaction terms find similar results to the fixed effects regressions without the interactions, providing a check on my results, and so my focus in the rest of this paper will be on the simple difference-in-difference and difference-in-difference with fixed effects results. 

I include below the regression coefficients the means of village mobile money use overall (12\%), the mean village mobile money use when at least one household uses mobile money (36\%), mean household mobile money use (18\%) and the mean shock (14\% for self reported shocks, 17\% for 1 standard deviation rainfall shocks).

The rainfall shock causes a drop in consumption of between 6\% and 11\% and is highly significant. Hence large rainfall shocks have a strongly negative effect on consumption per capita, confirming prediction 2 that in the absence of any mobile money use by the household or village, rainfall shocks have a negative effect on consumption.  

Having 36\% of the village using mobile money (the mean for villages where at least one person in the village uses mobile money), results in consumption per capita of other households in the village being between 4\% and 10\% higher and is highly significant in all but one of the results. Before household fixed effects are included, the impact of mobile money use on the using household is between 12 and 14\% and strongly significant. However, once household fixed effects have been included, this falls to a benefit of 3\% of consumption and is not significant. This confirms there is bias in the direct effect of mobile money use from a selection effect. Overall it can be seen that in the absence of an aggregate shock, users of mobile money share remittances with the village, resulting in per capita consumption of everyone in the village increasing. A user of mobile money gets no benefit above the benefit dependent on the proportion of households in the village using mobile money. 

Turning to the interactions of mobile money with the shock dummy; when an aggregate shock occurs, the village mobile money percentage is insignificant and sometimes negative. Hence households do not benefit from having other people using mobile money in the village when a rainfall shock occurs. In contrast the household mobile money use interaction with the aggregate shock is positive and significant, varying between 8\% and 14\%. When a rainfall shock occurs the household using mobile money no longer experiences a drop in consumption and may even get a slight increase in consumption. 

In my model, this corresponds to case 3 where mobile money users are choosing not to share remittances when an aggregate shock occurs, instead choosing to smooth their own household consumption instead. The positive and significant coefficient on the percentage of the village using mobile money suggests that when an aggregate shock has not occurred households are sharing remittances with others in their village, indicating households are pooling resources. This is a new result in the literature on mobile money, indicating mobile money services have wider benefits than just to the user. However, when a negative aggregate shock occurs users keep any increase in remittances for themselves and do not give more to the rest of the village to help others smooth the shock. There are many potential explanations for this result which can be seen by looking at risk sharing as a repeated game, by hidden information or by traditional norms of the village. I explore these in detail in the Conclusion.   

